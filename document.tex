%%% template.tex
%%%
%%% This LaTeX source document can be used as the basis for your technical
%%% paper or abstract. Regardless of the length of your document, the commands
%%% are all the same.
%%%
%%% The "\documentclass" command is the first command in your file. If you want to
%%% prepare a version of your article with line numbers - a "review" version -
%%% include the "review" parameter:
%%%    \documentclass[review]{acmsiggraph}
%%%

\documentclass[review]{acmsiggraph}

\usepackage{lineno}
\usepackage{float}
\usepackage{multirow}
\usepackage{array}
% \usepackage{units}
% \usepackage{color}

\usepackage{wrapfig}
\usepackage{tabulary}
\usepackage{amsmath} % assumes amsmath package installed
\usepackage{amssymb}  % assumes amsmath package installed
\usepackage{algorithm}
\usepackage[noend]{algpseudocode}



\algdef{SE}[DOWHILE]{Do}{doWhile}{\algorithmicdo}[1]{\algorithmicwhile\ #1}%


\newcommand{\todo}[1]{\textcolor{red}{TODO: #1}}

%%% Title of your article or abstract.

\title{Participatory Layout Design-Fabrication with Native Forms of Natural Materials}

\author{Hironori Yoshida\thanks{e-mail:hyoshida@hy-ma.com}   Takeo Igarashi\thanks{e-mail:takeo@acm.com}, Maria K Larsson, Masaaki Miki
 \\www.hy-ma.com, the University of Tokyo
}



\pdfauthor{Hironori Yoshida}

%%% Used by the ``review'' variation; the online ID will be printed on
%%% every page of the content.

\TOGonlineid{45678}

% User-generated keywords.

\keywords{radiosity, global illumination, constant time}

% With the "\setcopyright" command the appropriate rights management text will be added
% to your document.

%\setcopyright{none}
%\setcopyright{acmcopyright}
%\setcopyright{acmlicensed}
\setcopyright{rightsretained}
%\setcopyright{usgov}
%\setcopyright{usgovmixed}
%\setcopyright{cagov}
%\setcopyright{cagovmixed}
%\setcopyright{rightsretained}

% The year of publication in the "\copyrightyear" command.

\copyrightyear{2016}

%%% Conference information, from the completed rights management form.
%%% The "\conferenceinfo" command has two parameters:
%%%    - conference name
%%%    - conference date and location
%%% The "\isbn" field includes the year and month after the article ISBN.

\conferenceinfo{SIGGRAPH 2016 Posters}{July 24-28, 2016, Anaheim, CA}
\isbn{978-1-4503-ABCD-E/16/07}
\doi{http://doi.acm.org/10.1145/9999997.9999999}

\begin{document}

%%% This is the ``teaser'' command, which puts an figure, centered, below
%%% the title and author information, and above the body of the content.

\teaser{
    \centerline{
       \includegraphics[height=0.23\paperheight]{images/fabrication/teaser1.png}
       \hspace{1mm}
         \includegraphics[height = 0.23\paperheight]{images/fabrication/teaser2.png}
 }
  \caption{Left: an overview of the workflow: 1.fix branches on plates. 2.scan the plates and upload the model. 3.play the game with scanned branches. 4. fabricate joineries by a CNC router. Right top: branch layouts partially designed by participants in a workshop. Right bottom: the fabricated 2D screen fence (2m x 0.9m). }

  \label{fig:teaser}
}


\maketitle

\begin{abstract}
Diverse natural materials such as stones and woods have been used as architectural elements since primitive shelters, however, the use of them in modern buildings is limited, mainly due to the irregular properties of non-standard materials.
In this paper, we take the diversity as playful inputs for design task, and present our game based design and fabrication platform for non-experts.
Taking tree branches as case study of non-standard found materials, our online game \textit{BranchConnect} enables users to design 2D networks of branches and realize the design with parametrically generated joineries.
As the generated design is realizable with ordinal 3-axis CNC milling machines, each connection has a customized unique joinery adapted to the diverse branch shapes.
The scoring system of the game guides users to design structurally sounding solutions with given branches.
Together with low-cost mobile scanning devices, everyone can contribute to design and fabrication process not only as a game user, but also from collecting branches around their physical environments and uploading them to our online platform.
For validating our process, we conducted a workshop with non-experts, and let them collect branches in a nearby forest, and design/fabricate a 2D fence by integrating multiple design solutions.
\end{abstract}

\linenumbers

%
% The code below should be generated by the tool at
% http://dl.acm.org/ccs.cfm
% Please copy and paste the code instead of the example below.
%
\begin{CCSXML}
<ccs2012>
<concept>
<concept_id>10010147.10010371.10010382</concept_id>
<concept_desc>Computing methodologies~Image manipulation</concept_desc>
<concept_significance>500</concept_significance>
</concept>
<concept>
<concept_id>10010147.10010371.10010382.10010236</concept_id>
<concept_desc>Computing methodologies~Computational photography</concept_desc>
<concept_significance>300</concept_significance>
</concept>
</ccs2012>
\end{CCSXML}

\ccsdesc[500]{Computing methodologies~Image manipulation}
\ccsdesc[300]{Computing methodologies~Computational photography}

\keywordlist

\conceptlist

\printcopyright

\section{Introduction}
Modern buildings are characterized by its uniformity; built upon the same principle of construction system, consisting of standardized building component and its assembly process.
The standardized construction system is favored because of its efficiency in design and production.
As each component satisfies specified structural performance, the resulting structure can be analyzed systematically.
On the other hand, the excessive standardization caused generic designs of architecture which are criticized for being detached from local culture ~\cite[frampton1997towards].
Reacting on the issue, designers and architects actively use local materials not only as inspirational sources, but also as a catalyst of their design to the local context \cite{oliver1997encyclopedia}.

% As a result, we see similar-looking buildings all over the world.
%The aesthetic quality of the combination of construction techniques to deal with building components is described as "techtonik".
% Including management of construction, guidelines for maintenance to building materials and construction techniques, we call it as building starndard.
Since primitive shelters and huts, traditional construction has used locally found natural materials directly in their native forms \cite{weston2003materials}.
Such a direct use can not compete with highly standardized materials and construction system, however, the uniqueness of native forms is a valuable quality which is lacking in standardized materials.
As the material is locally obtained, building and living get much closer, thus people using the building can easily commit design and fabrication, fostering the sense of belonging to the community.
Locally obtained materials can easily connect design and the context of built environment in this way, but their irregular properties limit the use of them in modern buildings.
Traditionally, craftsman has taken care irregular natural materials varying their native forms and dynamically design the global design by considering individual material properties \cite{pye1968nature}.
Such a task is difficult to be automated and these skills are developed through years of training, thus the use of native forms typically inaccessible for end users and costs more than standardized construction system. \\

This paper aims to make the above-mentioned qualities of materials in native forms more accessible for end users by leveraging digital technologies.
We use locally obtained branches which can be found almost everywhere not only in countryside but also in urban environment such as parks and along streets.
Public service takes care of these branches: annual pruning, storing, and chipping or burning with some costs.
The size of branches (from 50 -300 $mm$ in diameter) is too small for furniture or other structural applications as building components.
It is a challenge for digital design and fabrication to utilize the diverse branches in meaningful ways.
High-precision but low-cost scanning devices and personal digital fabrication machines make it possible to analyze and control natural materials in diverse native forms.

The key technical difficulty of materials with natives forms is design.
A possible approach is optimization: minimize an energy function which integrates structural and fabrication costs.
This approach, however, is limited to particular design scenario with specific materials.
Furthermore, the concept of optimum solution is well suited to goals such as efficiency or low-cost, but these goals are not the qualities materials in natives forms can compete with standardized construction systems.
Instead, we take humans in our scan-design-fabricate workflow not only to solve the design problem, but also to provide an opportunity for people to participate in the workflow.
Traditionally, in case of constructing public and symbolic buildings, such as church, people in a community took initiatives from fund raising to design, or even in construction process.
% In this way, our method would be applicable to other kinds of materials with diverse shapes.

% The humans-in-the-loop also enrich the connection between users and a building.
% put emphasis on the forementioned qualities such as diverse solutions not only from the native diverse shapes but also from multiple solutions by people living and using the local community.
% In modern architecture, user partricipation in design has been commonly explored for enhancing connection between the design and the context as a reaction against the generic buildings.

%Historically, local natural stones and wavy trees have been commonly used to construct primitive huts and shelters \cite{weston2003materials}.
% Over the years, each unique solution has been converged into a symbolic building style per community, described as "vernacular" in architecture


In this paper, we report our case study to design and fabricate an architectural element out of irregularly shaped branches, using our humans-in-the-loop system.
We developed an online platform where users can post branches found at hand, and design with them through the online game \textit{BranchConnect} \footnote{Please visit and play the game. \url{https://branchconnect.herokuapp.com/start}}.
The game system itself helps users to design feasible branch layouts and enable them to fabricate customized joints to connect them together without screws and adhesives.
The design of our joint extends the traditional orthogonal lap-joints to various angles within a range, freeing the diverse forms of woods from orthogonal connections.
Physically collected branches are digitally scanned and stored in a cloud database \textit{BranchCollect}, and offline application \textit{Branch Importer} analyzes forms of branches and upload them to the database.
The simple visual feedback and scoring system of the game guide users to feasible solutions, which are further inspected by an offline application \textit{G-Code\footnote{G-Code is the generic name for a control language for CNC machines.} Generator} for CNC milling process.
The game system and developed import/export applications are currently limited to branches, however, the principle of human-in-the-loop with design is applicable for other kinds of materials with diverse irregular forms.
We hope our method sheds lights on materials such as waste from demolition of buildings for various design applications.

% but selected  as well as fabricating differentiated joints by a CNC router.
% We propose an accessible design and fabrication platform, which allows multiple users to design with diverse irrefular forms of branches.

In summary, our contributions are
\begin{itemize}
 \item{a workflow enabling to take natural materials in native forms as design components.}
 \item{an online game-based approach to participatory architectural design and fabrication.}
 \item{a method to design and fabricate customized non-orthogonal joints using high-resolution contours.}
\end{itemize}



% First of all, non-standard materials are considered to be expensive and less performative compared to standardized building materials.
% For example, fabricating wooden joints is considered to be an art due to the heterogeneous wooden structure and its precisions, resulting in higher cost than repetitive assembly of standard components \cite{seike1977art}.
% In the context of fabrication, Pye \shortcite{pye1968nature} described such a skill as ``\textit{workmanship with risks}'', in contrast to ``\textit{workmanship with certainty}''.
% Due to the complexity, these skills are not fully automated yet, thus we rely on skilled-workers to design and fabricate with these diverse materials.
% Second of all, building design and construction became highly specialized professions.
% Not only skills and knowledge, but also designers and architects must communicate with local municiparity to clear legal building codes.
% Although participatory design has been common in architectural design process, the degree of participation is limited.
% There is no such a platform to include local community to design and construction.

% In this project, we put emphasis not only on sustainable, economical aspects of architecture, but also on cultural, communal aspects.
% With the digital technology, we make the design process accessible for users by online browser game.
% The approach is similar to participatory design process in architecture.
% We believe that the use of non-standard materials provides an alternative design approach to modern architectural design.
% As a prospect, we also aim to reveal implicit knowledge to handle such non-standard materials, which has been fostered in craftsmanship.

\section{Related Work}
3D printers and CNC routers made digital fabrication more accessible, and pre-fabricated customized parts are often used in buildings nowadays \cite{knaack2012prefabricated}.
According to Pye, these parts are processed from highly standardized material, thus its digital fabrication process is ``workmanship with certainty''; a batch process of reading G-Code and strict execution of the code.
On the other hand, as ``workmanship with risks'' with digital technologies, interactive fabrication enables machines to pick up uncertain happenings and react on it \cite{willis2011interactive}.
Mueller developed interactive laser cutting, taking user inputs and recognizing placed objects in a fabrication scene \cite{Mueller:2012:ICI:2380116.2380191}.
While their system interprets objects as simple platonic geometry, our work takes the branches with diverse shapes.
While Crowdsourced fabrication project took advantage of humans-in-the-loop in their fabrication system \cite{lafreniere2016crowdsourced}, our work puts emphasis on crowd-sourced design.
As a crowd-sourced design system, Jerry et. al., developed a platform for light users to design trees and plants \cite{talton2009exploratory}.
Our design process is not parametric modeling, also directly linked to physical world. \todo{ill-logic}

There are few works that take natural materials with native shapes as design components.
Schindler and his colleagues used digitally scanned wood branches and used them for furniture and interior design elements \cite{schindler2014processing}.
Monier and colleagues virtually generated irregularly shaped branch-like components and explored designs of large scale structure ~\cite{monier2013use}.
Using larger shaped forked tree trunks, \textit{Wood Barn} project designed and fabricated custom joineries to construct a truss-like structure\cite{woodbarn}.
\textit{Smart Scrap} project digitally measured lime stone leftover slates from a quarry and digitally generated assembly pattern of slates \cite{smartscrap}.\\

In industry, recognition of irregularly shaped objects is essential for waste management.
\textit{ZenRobotics} developed a system that sorts construction and demolition waste by picking objects on a conveyor belt using robotic hands \cite{lukka2014zenrobotics}.
For factory automation purpose, there is a system that recognizes irregularly shaped objects and sort them into a container \cite{sujan2000design}.
Getting out from factories, autonomous robotics in construction site is a hot topic among roboticists \cite{feng2014towards}.
\textit{In-situ Fabricator} is a system which could be installed in construction site and co-operated with human workers \cite{dorfler2016mobile}.
Once robot is autonomously localize itself in such an environment, it can build foundational structure for further construction \cite{napp2014distributed}.
Using locally found objects on-site, such a system can be much simpler.\\

While these projects demonstrated the capability of digital fabrication processes to handle irregularly shaped materials, design process is still dependent on a designer or architect who has experiences with materials or has access to special software.

Cimerman discussed architectural design practices that took computer-mediated participatory design \cite{cimerman2000participatory}.
He mentioned three motivations of digital participatory design:
\begin{itemize}
 \item{Including stakeholders in creation of one's environment.}
 \item{Experimenting diverse design tastes from multiple point of views.}
 \item{Solving complex design tasks with full of diverse solutions.}
\end{itemize}
Opening database of available local materials, people with various backgrounds can involve in design process, which could lower the cost of design fee with natural materials with native shapes.
For example, \textit{Nano-Doc} took gamification approach to search valid nano-particle designs against tumors out of infinite design space \cite{hauertcrowdsourcing}.
\textit{DrawAFriend} has developed an online game to collect big-data for drawing applications which assist humans with auto-stroke assistance \cite{limpaecher2013real}.
While these works developed games which are collecting data for solving medical or engineering problems, our game intends to provide a participatory design platform as a solution for social and cultural problems context-aware architectural design with locally found natural materials.
 \todo{ambiguous!}

\section{Workflow}
As illustrated in the left of Figure \ref{fig:teaser}, our workflow starts from physically collecting branches.
The collected branches are uploaded to cloud database by \textit{Branch Importer}, and served to online game-based design application \textit{BranchConnect}.
The game is working on browsers and accessible from laptop computers and mobile devices.
As in the right of \ref{fig:teaser}, users can explore a global design with multiple branch layouts.
Once the global design is fixed, designed layouts are further inspected by \textit{G-Code Generator}, which generates customized joineries for CNC milling.
After finishing the milling process, users physically assemble branches and complete the fabrication process.
The pipeline of the workflow is illustrated in the Figure \ref{fig:pipeline}.
In this section, we describe three steps in the pipeline: Digital Model Acquisition, the Game System, and Fabrication.

\begin{figure}[ht]
  \begin{center}
    \includegraphics[width = 0.4\paperwidth]{images/workflow/pipeline.png}
    \caption{A pipeline from model acquisition to fabrication.}
    \label{fig:pipeline}
  \end{center}
\end{figure}

\subsection{Digital Model Acquisition}
Our system takes textured mesh model or point cloud with colored vertices.
As complete mesh model provides more robust results with 3D shapes of branches, we describe our process based on mesh model as an input.
There are various methods and software available for scanning 3D models.
As for scanning setup, we describe in the Section \ref{sec:casestudy}.
Taking mesh model with colored texture, our \textit{Branch Importer} provides functions such as object detection, skeleton extraction, branch type classification, and fixture point setting.

The scanned result is a mesh model representing branches with a fixed plate.
The system first idendifies branches by applying simple height threshold, and then applies contour detection (we currently use \textit{findContour} in OpenCV \footnote{Open Source Computer Vision Library: \url{http://opencv.org/} }).
The obtained 2D contours are used for extracting skeletons and clustering point cloud in the mesh model.
Contours are triangulated and skeleton points are extracted from middle points on edges of triangles.
These middle points are compared with top view image from OpenCV.
If the point is inside of a contour, the middle point is counted as a valid point.
After extracting valid middle points, the connectivity of skeletons is analyzed. 
In case grafting branch is detected, a new skeleton sub-branch is added. 
The result is shown in Figure ~\ref{fig:skeleton}.
%Evaluating the number of sub-branches, the branch is morphologically classified.
Metal fixture locations are confirmed by simple mouse-clicks and set as invalid points.
The acquired information is stored in a cloud database.

\begin{figure}[ht]
  \includegraphics[width = 0.4\paperwidth]{images/importer/importer.png}
  \caption{An interface of \textit{Branch Importer}. Left: a top ortho-view image of textured mesh model. Right: Extracted skeletons are shown with blue dots. The beginning of skeletons is shown bigger dots, and the red dots are invalid points defined by a user. }
  \label{fig:skeleton}
\end{figure}


\subsection{BranchConnect: The Game}
The online game is accessible by laptops and mobile touch devices, and many users can play at the same time.
The objective of the game is to collect valid layouts of branches which are fabricatable with 3 axis CNC milling machines.
By analyzing the connectivity of branches and target points, the game checks structural feasibility of a given layout. 
The guidance system during furniture design inspected connectivity, durability, and stability \cite{umetani2012guided}. 
Unlike their work, our game focuses on fabricatability with minimum consideration of gemetric connectivity due to the limited resources with mobile devices.


%The workflow of the game is illustrated in Figure~\ref{fig:game_flowchart}.
%\begin{figure}[ht]
%  \begin{center}
%    \includegraphics[width = 0.25\paperwidth]{images/system/systemFlowChart.pdf}
%    \caption{The workflow of \textit{BranchConnect}. Branch data and user design data are stored on cloud database.}
%    \label{fig:game_flowchart}
%  \end{center}
%\end{figure}

Firstly a user selects a frame indicating multiple target points to be connected, and then selects a set of branches fixed on a plate (The left in Figure \ref{fig:game_interface} ).
After selecting the target frame and the set of branches, the user is guided to the game interface, consisting of the frame with the target points, and the set of available branches (The right in Figure \ref{fig:game_interface} ).
The user picks a branch from the available set on the bottom, and then places it in an arbitrary 2D pose through basic manipulations such as move, rotate, and mirror.
The number of available branches differs depending on plates.
With the feasible diameter of branches (over 2cm) and the plate size (50cm x 50cm), the number of available branches is most likely up to six.
Within the limited number of branches, the user bridges all the target points by connecting all the used branches in one group.
The game is completed when all the target points are connected.
For higher score, the user can modify the design after the completion, and save it to the database.

\begin{figure}[H]
  \begin{center}
    \includegraphics[width = 0.4\paperwidth]{images/interface/game_interface.png}
    \caption{Left: the selection interface for target frames (top) and branch panels (bottom). Right: the start interface of the game.}
    \label{fig:game_interface}
  \end{center}
\end{figure}
%
\subsubsection*{System Overview}
There are many collision detection libraries available, however, our game needs to detect intersected branch pairs, thus surface contact detection is overkill for our application.
Also most branches come with free-form concave shapes, thus further geometric preparation such as convex decomposition is necessary for using these libraries.
For fast and robust intersection detection, our system extensively uses skeletons of branches.

Hubbard and Philip developed collision detection by representing object with hierarchical 3D spheres aligned on skeletons \cite{Hubbard:1996:APS:231731.231732}.
Our system takes similar approach but limited in 2D, but more focused on searching fabricatable joints.
In the game, simplified skeletons are used to find the pair of closest skeleton points between two branches.
When a branch is selected, it is counted as an active, and the system searches the closest skeleton point from skeletons of other branches.
More precise joint calculation with higher resolution is further described in the next section.

A joint is created when an intersecting pair is detected, and the pair forms a group.
The group is used for evaluating connections between target points (\textit{Bridged}).
The game is completed once all the target points are connected by a group of branches.
The conditions of joint and group are indicated with simple color-code.
Once the user finishes positioning, score is updated with weighted sum of parameters.
Together with the color-code, the score update guides the user to form a valid design.


\subsubsection*{Joint Condition}
Joint is the essential entity not only in the game but also in the fabrication process of customized lapped joineries.
Each fabricated joint works as a rigid joint, and we do not calculate structural performance of each joint.
Figure~\ref{fig:joint_condition} illustrates valid and invalid joint conditions.
Our joint only takes crossed pair (see Figure~\ref{fig:joint_condition}.1) because they are structurally stable, relatively simple to fabricate, and creates diverse designs. 
Tangential connections are counted as invalid as fabrication of tangential joinery is challenging with small branches (see Figure~\ref{fig:joint_condition}.3).

\begin{figure}[ht]
	\begin{center}
		\includegraphics[width = 0.4\paperwidth]{images/system/joint_conditions_2.png}
		\caption{Joint conditions. 1.valid joint. 2.invalid for violating the angle. 3.invalid tangential connection. 4.invalid for connecting on a fixture point. }
		\label{fig:joint_condition}
	\end{center}
\end{figure}


A valid joint's angle stays within a fixed range (see Figure~\ref{fig:joint_condition}.1 and 2). 
Joints close to metal fixtures are also counted as invalid (see Figure~\ref{fig:joint_condition}.4).
Valid and invalid joints are displayed with green and red respectively.
When a branch is connected to a target point, the specially added score is deisplayed in a pop-up square, also the graphics of target point's and the branch are changed.
The branch is trimmed at the connection point with the target point, and the trimmed length is subtracted from the score.

To describe the process, we let $\mathcal{B}$ denotes a set of all the available branches, and each branch as $ b_i \in \mathcal{B}$.
While the process checks joint condition through all the branches $\mathcal{B}$, each detected joint is stored in each branch $b_i$, categorized in different conditions such as valid and invalid joints denoted as $j_{\text{valid}, j, i} \in \mathcal{J}_{\text{valid},i}$ and $j_{\text{invalid}, j, i} \in \mathcal{J}_{\text{invalid},i}$ respectively.
When a branch $b_i$ is connected to one of target point $t_j \in \mathcal{T} $, the $t_j$ is stored in $b_i$.


A flowchart of the game system with joint and group conditions is illustrated in Figure \ref{fig:system_flowchart}.

\begin{figure}[ht]
	\begin{center}
		\includegraphics[width = 0.35\paperwidth]{images/system/closestPointAlgorithm.pdf}
		\caption{Left: an overview of the game system with 1.joint checkand 2.group check and 3. score calculation. This process is iteratively executed while a user is exploring layout by dragging a branch. The joint check process is further illustrated in the right, and group condition check is described in Algorithm \ref{al:connection}. }
		\label{fig:system_flowchart}
	\end{center}
\end{figure}

% well as the paired branch $b_{\text{paired},j} \in \mathcal{P}_{\text{paired},i}}$.



\subsubsection*{Group Condition}

After checking joint conditions of all the pairs of branches, the system checks the number of groups as well as its connection with the target points on a frame.
If a group is not connected to any target point nor other groups, the group is \textit{Islanded} and structurally invalid.
While a user is positioning a branch by dragging or rotating, groups are continuously calculated and indicated by simple color (Figure \ref{fig:group}).

\begin{figure}[ht]
  \begin{center}
    \includegraphics[width = 0.4\paperwidth]{images/interface/groups.jpg}
    \caption{Left: valid group with two target points connected. Middle: valid but three groups. Right: invalid due to the \textit{Islanded} situation. }
    \label{fig:group}
  \end{center}
\end{figure}


After all the joint conditions are checked, we evaluate group conditions.
Through checking the all the branches $\mathcal{B}$, the first group $g_0$ is created and stored as $b_0$.
%The other paired branch $b_{\text{paired},i}$ stored in $j_{\text{valid}, j}$ is used for tracing the connection with other paired branch in each valid joint $j_i$. \todo{incomplete sentence!}
The game is completed when the number of $\mathcal{G}$ is one, and all the target points are connected with the group.
The algorithm chekcing group conditions is described in \ref{al:connection}.

\begin{algorithm}
  \caption{Group Condition Update Algorithm}
  \begin{algorithmic}[1]
    \Function{UpdateGroups}{$\mathcal{B}$}
    \State{Reset all the groups $\mathcal{G}$ }
    \State{Create new group $g_0$}
    \State{$b_0 \text{ is added to } g_0$}
    \If{$b_0$ has connected target point $t_i \in \mathcal{T}$}
      \State{$g_0 \text{ sets } t_i$}
    \EndIf
    \State{$g_0 \text{ is added to }  \mathcal{G} $}

    \For{each branch $b_i$ in $\mathcal{B}$}
    \State{\textit{GroupConnection} $\gets false$}
      \For{each group $g_j$ in $\mathcal{G}$}
        \For{each branch $b_j$ in $g_j$}
          \If{ $b_{\text{paired},i} \in \mathcal{P_i}$ has $b_j$}
            \State{ $b_i  \text{ is added to } g_j $}
            \State{ \textit{GroupConnection}  $\gets true$}
            \If{($b_i$ has $t_i$) and ($g_j$ has $t_j$) }
              \State{ Set $g_j$ as \textit{Bridged}}
            \EndIf
            \If{$g_j$ has no $t_j$}
              \State{ Set $g_j$ as \textit{Islanded}}
            \EndIf
            \State {\textit{break}}
          \EndIf
        \EndFor
      \EndFor

      \If{ \textit{GroupConnection} is \textit{false} }
        \State{create new group $g_{new}$}
        \State{ $b_i  \text{ is added to } g_{new}$}
        \State{$g_{new} \text{ is added to }  \mathcal{G}$}
      \EndIf
    \EndFor

  \EndFunction
  \end{algorithmic}
  \label{al:connection}
\end{algorithm}

\subsubsection*{Score Calculation}
In the score calculation processes, follwoing entities forms the score: the numbers of valid and invalid joints on each branch, the number of groups as $N(\mathcal{G} )$, the number of islanded groups as $N(g_{\text{islanded}} \in \mathcal{G} )$, the number of bridged target points as $N(t_{\text{bridged}, i}) \in \mathcal{T} )$.
The trimmed lengths of branches which are connected with target points are denoted as $trimmed(t_j, b_i)$
The score is weighted sum of these joint and group conditions, denoted in Equation (\ref{eq:cost}).
The weights 'w1...w5' are non-negative weight coefficients pre-adjusted in advance by authors.




\begin{equation} \label{eq:cost}
 \begin{aligned}
 Score =  &\; w_1  \sum_{1}^{N(\mathcal{B})} \sum_{1}^{N(\mathcal{J}_{\text{valid},i})} j_{\text{valid}, j, i} 
	& + \; &w_2  \sum_{1}^{N(\mathcal{B})} \sum_{1}^{N(\mathcal{J}_{\text{invalid},i})} j_{\text{invalid}, j, i}\\
+ &\; w_3  \sum_{1}^{N(\mathcal{G})} g_{\text{islanded}} \;
	& + \; &w_4  \sum_{1}^{N(\mathcal{T})} t_{\text{bridged}} \; \\
+ &\; w_5 \sum_{1}^{N(\mathcal{T})} trimmed(t_j, b_i)
 \\
   \textrm{s.t.} \; w_j  \geq  &\;0 \; \forall j \in 1, \dotsc , 5 
 \end{aligned}
\end{equation}

\subsection{Fabrication}
After a design is selected for fabrication, the validity of the design is further inspected with a high-resolution model.
The \textit{G-Code Generator} was developed for fine-tuning the design by checking real-time feedback of updated joineries on branches with scanned orientations (see Figure \ref{fig:gcode_gen}).
Users can change fabrication parameters such as offset ratio for the side cuts, milling bit diameter, overlapping ratio for defining the center cut depth, feed-rate, moving height and so forth.
After confirming the fabrication settings and milling paths, it generates G-Code.

\begin{figure}[ht]
  \begin{center}
    \includegraphics[width = 0.4\paperwidth]{images/system/joint_generator_2.png}
    \caption{Interface of \textit{G-Code Generator}. Users can tweak the design on the left side and immediately see the updated joints and milling paths on the right. }
    \label{fig:gcode_gen}
  \end{center}
\end{figure}

Some fabrication factors such as mirror and invalid points are already considered by \textit{Branch Importer} and the game system.
In this section, we describe the process of joinery generation.
Each joinery geometry is parametrically modeled with: two plane surfaces on the sides of branches and one plane top surface (see Figure \ref{fig:joint_geometry}).
The geometry creates rigid connection with the irregularly shaped sections of branches.\\

Similar to the joint searching process with skeletons, the \textit{G-Code Generator} searches a set of four closest points from high-resolution contours, expecting that every intersected contour has four curves.
After finding the four closest points, it trims two curves from each contour of branch. (two from intersecting branch and two from intersected branch) at each joint.
The trimmed contours are transformed to the original scanned orientation and used for generating milling paths.
Two curves from an intersected branch are used for side cuts milling paths, which are inwardly offset paths of the original branch contours. 
The center cuts are paths that are plaining the top surface of the joint.
Height of center cuts is calculated with a diameter of cross-section at the center of a joint and the actual height information stored in point cloud. \todo{strange sentence}


\begin{figure}[ht]
  \begin{center}
    \includegraphics[width = 0.4\paperwidth]{images/system/joint_milling_diagram_4.png}
    \caption{An example of intersected pair: 1. an assembled pair of branches. 2. branches after the center and side cuts are milled 3. left: a layout defined by a user right: the original orientations of branches with generated milling paths with red color.  }
    \label{fig:joint_geometry}
  \end{center}
\end{figure}



\section{Experiment}
A design and fabrication workshop was organized to examine the validity of our system with a specific design target and a location.
We selected a public community house where people in the community share the space and regularly use the facility.
Participants are from the community, mainly children aged 10 and under (4, 7, 9 , and 10 years old) and their parents (Figure~\ref{fig:workshop}).
We specifically selected children with the age range as non-experts without experiences in computational design or digital fabrication, also for observing the clarity and attractiveness of the game.

The goal for the participants is to contribute to design and fabrication process of a screen wall (2m by 0.9m) consisting of 8 rectangles.
Each rectangle has preset target connection points from 3-5 points.
Participants are asked to follow the entire process from collecting and fixing the branches on a plate, scanning the plate, complete designs by playing the game, and assembly after the CNC milling.
% Regarding the game, xx solutions are given by participants and the total duration spent for the game was xx.
Participants were informed about the goal of the workshop, and each process was introduced by experienced tutors.

\begin{figure}[ht]
  \begin{center}
    \includegraphics[width = 0.4\paperwidth]{images/fabrication/workshop_setup.png}
    \caption{An overview of the workshop. 1.the overview of the space. 2.cutting 3.preparation 4.scanning 5.retouching 6.playing the game 7.CNC milling.}
    \label{fig:workshop}
  \end{center}
\end{figure}

\subsection{Setup}
\subsubsection*{System and Hardware Setup}
We used two iPad minis with iSense scanners attached for scanning branches, and a 3 axis CNC milling router with a 6 mm diameter milling bit.
We used a laptop PC (Lenovo w240 with intel core i7) for running \textit{Branch Importer} and \textit{G-Code generator}, as well as operating the milling machine.
The scan area of iSense is 500mm x 500mm, and the milling machine's stroke along z-axis is 70mm, which provide geometric constraints for available branch sizes.
\textit{BranchConnect} was hosted at \textit{Heroku} cloud server \footnote{ Heroku is a platform as a service (PaaS) that enables developers to build, run, and operate applications entirely in the cloud. \url{https://www.heroku.com/}},
and we used \textit{MongoDB} \footnote{ MongoDB is a free and open-source cross-platform document-oriented database program. \url{https://www.mongodb.com/}} as a cloud database.

\subsubsection*{Preparation}
The participants were asked to collect branches with 2 - 10 cm in diameter.
The lower bound of the diameter is for milling process would not destroy them, and the upper bound is for the limit of z-stroke of the CNC router.
The collected branches are cut in arbitrary lengths, not longer than 500 mm due to the limit of scanning area. \\

As our game system and fabrication process take 3D branch shapes as 2D contours with height map, these constraints work positive for the system automatically filtering out branches with large 3D twists.
We ask participants to scan with iPad + iSense and prepare feasible mesh model by themselves.
After obtaining mesh models, tutors import models from iPads to a laptop and upload them to database by \textit{Branch Importer}.


\subsubsection*{User Experiences with the System}
After models were uploaded to the server, users can see that their plates are added in the selectable branch plates with their names and locations.
Users can access to the start page by PC and mobile devices.
We prepare both options and let participants choose a device.

A user is firstly directed to a start page and asked to submit a user name.
Secondly, the user is navigated to target frame selection page, and asked to pick one out of eight frames.
Each frame has different target points.
The interface also shows the completed branch organizations within each target frame.
If there are multiple designs, three designs with highest scores are displayed.
The user can change the currently displayed design by clicking within each frame and choose either starting their design from scratch, or select the design and improve it.

After selecting a target frame, the user goes to branch selection page, displaying 15 plates when the workshop was held.
In this page, they see the plates made by themselves on the page, as well as their names on the plate.
The user can select the same plate for designing other target frames.
By clicking a displayed branch plate, the user is navigated to the game interface.
After completing to bridge all the target points, the design is automatically uploaded to the database, but the use can continue to design.
Tutors and participants select design solution for each target frame, and an experienced tutor operates \textit{G-Code Generator} as well as the CNC router.
Participants were asked to assembly branches after joineries were milled.

\subsection{Results}

\subsubsection*{Collecting Branches}
The diameter and length constraints for available branches worked as guidelines for participants rather than restricting finding and cutting arbitrary branches.
After cutting branches in certain lengths, participants fixed branches on plates by thin metal plates with screw holes.
It was straightforward for them to firmly fix branches so that they are not moved during milling process.
These fixture points are counted as invalid points in the game where joinery points can not be generated.
The participants built 3 plates with 3-6 branches fixed on each plate.

\subsubsection*{Model Acquisition}
iSense 3D scanners come with an intuitive software for scanning and modifying models.
After we gave an instruction, participants practiced several scans and successfully scanned models without any problem

Each scanning and re-touching took 2-3 minutes, and 30 seconds for generating data by \textit{Branch Importer}.
In total, we scanned 15 plates, 75 branches, and 35.3m in total length including sub-branches. \todo{check the length again!}
We got 40 $I$ shaped branches, 19 $V$ shaped branches, and 16 $Y$ shaped branches.
The result is shown in Figure \ref{fig:scannedplates}.

\begin{figure}[ht]
  \begin{center}
    \includegraphics[width = 0.4\paperwidth]{images/fabrication/all_plates.png}
    \caption{An overview of all the 15 scanned plates for the workshop. Top raw of each set shows ortho-top views of scanned mesh models, and the bottom raw is the recognized branches assigned random colors.}
    \label{fig:scannedplates}
  \end{center}
\end{figure}


\subsubsection*{Game}
Most participants used iPads for navigating pages and playing the game.
All the participants understood the goal immediately, however, they had difficulties with user interface, such as rotation and flipping operation by gestures.

One participant switched to play by a PC for more precise control due to the problem.
All the participants chose to develop their own designs from the scratch, although they had instruction about the "continue existing designs".

Although there is no time limit in the game system, we set 30 minutes for playing the game, and 8 solutions were given by participants.
xxx frames were completed per participants and xxx participants completed the whole eight target frames.
The average score is xxx, and average playing duration was xxx to complete each target frame.

\subsubsection*{Design Consensus \todo{this section might be removed}}
As the target frame selection page can display designs not only from one user but also from all the others at once, we could get an overview of design options.
The designs are displayed as score descending order but limited numbers, we could find feasible solutions with mostly all the target points were bridged.
As participants were excited by seeing their branches and designs, we took some of their solutions and their plates for the fabrication although their solutions did not satisfied completion of the game.

\subsubsection*{Fabrication}
We did not have major problems for converting designs to G-Code.
The main problem here was the accuracy of acquired contours.
We observed most of scanned models have occluded regions between plates and branches, which create interpolated faces during solidifying process, resulting in outwardly offsetted contours.
Six pairs of branches were loosely connected because the calculated contours were 2-3mm eroded than the actual sizes.
We avoided this problem by trimming branches from 2-5 mm higher than the plate surface.
After this operation, the rest of connections were tightly connected. \\

We also observed that many milling paths were 5-10 mm off from the center of planned joints.
Multiple reasons are assumed; such as deformation of mechanical parts of the CNC router, the resolution of acquired contours of branches, misaligned orientation of the plate compared to the scanned model and so forth.
We modified the \textit{G-Code Generator} so that an operator can freely adjust the absolute origin of the generated milling paths.
We usually set a marker around the center of the plate.
After this modification, the misalignment was improved.
We still observed misaligned joint center but it was 5 mm at the maximum.

\subsection{Discussions}
-multiple plates
-design decisions
-most of plates were already built
-object detection
-bad user interface
-misaligned branches, orientation detection, interactive fabrication

\subsubsection*{Interface}
We received several requests by participants regarding interfaces of the system.
\begin{\begin{itemize}
  \item {multiple plates for designing a target frame.}
  \item {designing freely without limiting the number of branches.}
  \item {iPad interface needs toggle buttons for keep a branch active.}
\end{itemize}}

\subsubsection*{Post tensioning by misaligned joints}
Branches can be elastically deformed thus they can absorb misaligned joint positions. Branches could absorb 3-5 mm misaligned joint positions due to the elasticity of the wooden materials.
The misaligned joint positions work as post-tensions in most cases, thus provided tightly fixed compositions.
We assume that this is only applicable when an applied bending moment and cut surface at a joint is orthogonal or not too much off from orthogonal.

\section{Conclusion}
% \subsubsection*{Summary}
In this paper, we presented a workflow to design and fabricate with branches in their native forms, which are not large enough to be used as standardized building components.
Our workflow was validated by the case study with lower-aged participants without design and fabrication experiences.

Our online platform with stored scanned branches and the game is accessible.
Multiple users can submit design layouts and explore a global design. % and locations.
Our branch joint and group condition update algorithms are running on the browser game which can be accessed from laptops and mobile devices, contributing to the accessibility of the presented workflow.
Together with the accessibility, the intuitive interface was simple for non-expert users, validated by the case study.
We successfully built a network of branches with rigid joints generated by our joint milling path generator.
Each of joints has customized lapped-joint geometry, which extends design possibilities of branches or woods in native forms.

% \subsubsection*{Limitations and Future Work}
Our workflow consists of many technological components such as skeleton extraction, structural optimization, object detection/recognition, and data-driven design-fabrication.
Focusing on the use of native forms, each step of our workflow has potential to contribute to each area with the use of native forms of natural materials.
Also, our workflow was developed based on the participation of users, thus the entire process is not necessarily automated, however, some tasks could be improved to assist users.

The skeleton extraction could take incomplete point-set directly from original tree branches before they are cut in length.
With data-driven approach, the system could distinguish trees and which part of tree the branch from.
With morphological analysis, the system could suggest users where to cut branches to achieve user-defined target design.
Structural and geometrical validity/invalidity of obtained materials could be analyzed.
Our workflow requires branches to be fixed on a plate, which takes the longest duration in the workflow except for in-between tasks such as moving and pausing.
Using a robotic manipulator with a gripper, the attaching process could be skipped.

Our game system is limited in 2D, whereas original branch forms have rich 3D geometry with textures.
In our case, these information was used in limited ways such as in skeleton extractions and G-Code generation.
Despite of successfully fabricated non-orthogonal joints, we did not complete the attaching branches to the target frames, as we prioritized to validate branch-branch joints.
Our layout design process is fully dependent on users with limited feedback during design process.
The game can provide suggestive feedback with structural analysis of each joint and entire structure.
% As the problem of limited number of branches and fixed target points to be connected, the system could assist humans to reach to structurally sound solutions with less efforts.

% The performance of our joint detection and alalgorithm could be
Our joint and group detection algorithms are limited with materials with skeletons, and our joint generator is limited to branches.
Both steps use down-sampled or high-resolution point sets.
It is valuable to validate the approach by comparing with other available methods such as collision detections or joint detection with down-sampled model by interpolation.

Finally, our game-based design could be applied to different purposes, not only for participatory layout design but also for collecting data of user behaviors during design.
Also application to other kinds of materials could be investigated.



\section*{Acknowledgements}
We appreciate the public house for hosting the CNC router and providing participants for the user study.
We thank to the film editor, Shin Yamane for shooting our video clips.
We also thank to the developer of an online game \textit{2048}, Gabriele Cirulli and his contributors, for sharing source code in Github community.


\bibliographystyle{acmsiggraph}
\nocite{*}
\bibliography{references}
\end{document}
