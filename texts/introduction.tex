\section{Introduction}
Modern buildings are characterized by its uniformity; built upon the same principle of construction system, consisting of standardized building component and its assembly process.
The standardized construction system is favored because of its efficiency in design and production.
Each component satisfies specified structural performance, thus an assembled structure can be analyzed systematically.
On the other hand, excessive standardization converged buildings into similar materials and details, resulting in the detached design from built environment.
Reacting on the issue, designers and architects actively use local materials not only as inspirational sources, but also as a catalyst of their design to the local context \cite{oliver1997encyclopedia}.

% As a result, we see similar-looking buildings all over the world.
%The aesthetic quality of the combination of construction techniques to deal with building components is described as "techtonik".
% Including management of construction, guidelines for maintenance to building materials and construction techniques, we call it as building starndard.
Since primitive shelters and huts, traditional construction has used locally found natural materials directly as they are found \cite{weston2003materials}.
Such a direct use can not compete with highly standardized materials and construction system, however, the uniqueness of native forms is a valuable quality which is lacking in standardized materials.
As the material is locally obtained, building and living get much closer, thus people using the building can easily commit design and fabrication, fostering the sense of belonging to the community \cite{}. \todo{citation}
Locally obtained materials can easily connect design and the context of built environment in this way, but their irregular properties limit the use of them in modern buildings.
Traditionally, craftsman has taken care irregular natural materials varying their native forms and dynamically design the global design by considering individual material properties \cite{pye1968nature}.
Such a task is difficult to be automated and these skills are developed through years of training, thus the use of native forms typically costs more than standardized construction system. \\

This paper aims to make the above-mentioned qualities of materials with native forms more accessible by digital technologies.
We use locally obtained branches which can be found almost everywhere not only in country-side but also in urban environment such as parks and along streets.
Public service takes care of these branches: annual pruning, storing, and chipping or burning with some costs.
The size of branches (from 5 -30 cm in diameter) is too small for furniture or other structural applications.
It is a challenge for digital design and fabrication to utilize the diverse branches in meaningful ways.
High-precision but low-cost scanning devices and personal digital fabrication machines make it possible to analyze and control natural materials with diverse native forms.

The key technical difficulty of materials with natives forms is design.
Optimization is a straightforward approach: minimize an energy function which integrates structural and fabrication costs.
This approach, however, is limited to particular design scenario with specific materials.
Furthermore, the concept of optimum solution is well suited to goals such as efficiency or low-cost, but these goals are not the qualities materials with natives forms can compete with standardized construction systems.
Instead, we take humans in our scan-design-fabricate workflow not only to solve the design problem, but also to provide an opportunity for people to participate in the workflow.
Traditionally, in case of constructing public and symbolic buildings, such as church, people in a community took initiatives from fund raising to design, or even in construction process \cite{}. \todo{citation}
% In this way, our method would be applicable to other kinds of materials with diverse shapes.

% The humans-in-the-loop also enrich the connection between users and a building.
% put emphasis on the forementioned qualities such as diverse solutions not only from the native diverse shapes but also from multiple solutions by people living and using the local community.
% In modern architecture, user partricipation in design has been commonly explored for enhancing connection between the design and the context as a reaction against the generic buildings.

%Historically, local natural stones and wavy trees have been commonly used to construct primitive huts and shelters \cite{weston2003materials}.
% Over the years, each unique solution has been converged into a symbolic building style per community, described as "vernacular" in architecture


In this paper, we report our case study to design and fabricate an architectural element out of irregularly shaped branches, using our humans-in-the-loop system.
We developed an online platform where users can post branches found at hand, and design with them through the game \textit{BranchConnect}.
The game system itself helps users to design valid branch layouts and enable them to fabricate customized joineries to connect them together without screws and adhesives.
The design of our joinery extends the traditional orthogonal lap-joints to various angles within a range, freeing the diverse forms of woods from orthogonal connections.
Physically collected branches are digitally scanned and stored in a cloud database \textit{BranchCollect}, and offline application \textit{Branch Importer} analyzes forms of branches and upload to the database.
The simple visual feedback and scoring system of the game guide users to valid solutions, which are further inspected by an offline application \textit{G-Code Generator} for CNC milling process.
The game system and developed import/export applications are currently limited to branches, however, the principle of human-in-the-loop for design is applicable for other kinds of materials with diverse irregular forms.
We hope our method sheds lights on materials such as waste from demolition of buildings for various design applications.

% but selected  as well as fabricating differentiated joineries by a CNC router.
% We propose an accessible design and fabrication platform, which allows multiple users to design with diverse irrefular forms of branches.

In summary, our contributions are
\begin{itemize}
 \item{a workflow enabling to take natural materials with native forms as design components.}
 \item{an online game-based approach to participatory design-fabrication.}
 \item{an algorithm to generate customized non-orthogonal joineries which are fabricatable by a CNC router.}
\end{itemize}



% First of all, non-standard materials are considered to be expensive and less performative compared to standardized building materials.
% For example, fabricating wooden joineries is considered to be an art due to the heterogeneous wooden structure and its precisions, resulting in higher cost than repetitive assembly of standard components \cite{seike1977art}.
% In the context of fabrication, Pye \shortcite{pye1968nature} described such a skill as ``\textit{workmanship with risks}'', in contrast to ``\textit{workmanship with certainty}''.
% Due to the complexity, these skills are not fully automated yet, thus we rely on skilled-workers to design and fabricate with these diverse materials.
% Second of all, building design and construction became highly specialized professions.
% Not only skills and knowledge, but also designers and architects must communicate with local municiparity to clear legal building codes.
% Although participatory design has been common in architectural design process, the degree of participation is limited.
% There is no such a platform to include local community to design and construction.

% In this project, we put emphasis not only on sustainable, economical aspects of architecture, but also on cultural, communal aspects.
% With the digital technology, we make the design process accessible for users by online browser game.
% The approach is similar to participatory design process in architecture.
% We believe that the use of non-standard materials provides an alternative design approach to modern architectural design.
% As a prospect, we also aim to reveal implicit knowledge to handle such non-standard materials, which has been fostered in craftsmanship.
