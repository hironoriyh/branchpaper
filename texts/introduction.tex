\section{Introduction}
Modern buildings are characterized by its uniformity; built upon the same principle of construction system consisting of standardized building components and its assembly process.
The standardized construction system is favored because of efficiency in design and production.
% Another upside is  the structural performance of the whole building can be analyzed as each component and connection are standardized.
On the other hand, the excessive standardization induced similar styles of buildings which are criticized for being detached from local contexts ~\cite{frampton1983towards}, thus designers and architects actively use local materials preserving original forms as a catalyst of their design to the local context ~\cite{oliver1997encyclopedia}.
As the material is locally obtained, building represents a local community, fostering the sense of belonging ~\cite{weston2003materials}.
Traditionally, in case of constructing public and symbolic buildings, such as church, people in a community participated in fund raising or even in construction processes \cite{weston2003materials}.

Today, buildings with locally found materials are rare mainly because of economical issue.
As these materials are typically used by preserving their irregular properties, installation of them requires skilled craftsman for ensuring quality and structural validity, resulting in higher cost.
 % can not compete with the efficiency of standardized systems.
Pye ~\shortcite{pye1968nature} argued craftsmanship as skills which can adapt its fabrication process according to irregular properties of materials and environments, and described as ``workmanship with risks'' in contrast to `workmanship with certainty''.
In this way, materials with irregular properties, or natural materials to be simple, exist almost everywhere but if
Since such a task require years of training and difficult to be automated, the use of materials with irregular properties are favored but limited in modern living environment, such as sculpture or other decorative works.
Also the knowledge to process natural materials are not explicit, thus users can not easily install such materials even though materials themselves are accessible.

%  thus the use of native forms typically inaccessible for end users and costs more than standardized construction system.
% Therefore, the use of natural materials is favored but there is no systematic use to overcome

This paper aims to make the above-mentioned qualities of materials in native forms more accessible for end users by leveraging digital technologies.
We use locally obtained branches as they are everywhere.
Usually public service maintain them by pruning, storing, and chipping or burning with some costs.
The size of branches (from 50 -300 $mm$ in diameter) is too small for furniture or other structural applications as building components.
It is a challenge for digital design and fabrication to utilize diverse branches in meaningful ways.
Theoretically, the combination of low-cost scanning and digital fabrication machines can handle natural materials in diverse native forms.
The key challenge here is design.
A possible approach is optimization: minimize an energy function consisting of structural and fabrication costs.
This approach, however, is limited to particular design scenario with specific materials.
Furthermore, the concept of optimum solution is well suited to goals such as efficiency or low-cost, which are not the criteria materials in natives forms can compete with standardized construction systems.
Instead, we take humans in our scan-design-fabricate workflow not only to solve the design problem, but also to provide an opportunity for people to participate in the workflow.

In this paper, we report our case study to design and fabricate an architectural element out of irregularly shaped branches, using our humans-in-the-loop system.
An online platform was developed so that users can post branches found at hand and design with them through the online game \textit{BranchConnect} \footnote{The game URL:  \url{https://branchconnect.herokuapp.com/start}}.
The game system itself helps users to design feasible branch layouts and enable them to fabricate customized joints to connect them together without screws and adhesives.
The design of our joint extends the traditional orthogonal lap-joints to various angles within a range, freeing the diverse forms of woods from orthogonal connections.
Physically collected branches are digitally scanned and stored in a cloud database \textit{BranchCollect}, and offline application \textit{Branch Importer} analyzes forms of branches and upload them to the database.
The simple visual feedback and scoring system of the game guide users to feasible solutions, which are further inspected by an offline application \textit{G-Code\footnote{G-Code is the generic name for a control language for CNC machines.} Generator} for CNC milling process.
The game system and developed import/export applications are currently limited to branches, however, the principle of human-in-the-loop with design is applicable for other kinds of materials with diverse irregular forms.
We hope our method sheds lights on materials such as waste from demolition of buildings for various design applications.

In summary, our contributions are
\begin{itemize}
 \item{a workflow enabling to take natural materials in native forms as design components.}
 \item{an online game-based approach to participatory architectural design and fabrication.}
 \item{a method to design and fabricate customized non-orthogonal joints using high-resolution contours.}
\end{itemize}
