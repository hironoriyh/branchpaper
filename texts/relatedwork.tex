\section{Related Work}
3D printers and CNC routers made digital fabrication more accessible, and pre-fabricated customized parts are often used in buildings nowadays \cite{knaack2012prefabricated}.
According to Pye, these parts are processed from highly standardized material, thus its digital fabrication process is ``workmanship with certainty''; a batch process of reading G-Code and strict execution of the code.
On the other hand, as ``workmanship with risks'' with digital technologies, interactive fabrication enables machines to pick up uncertain happenings and react on it \cite{willis2011interactive}.
Mueller developed interactive laser cutting, taking user inputs and recognizing placed objects in a fabrication scene \cite{Mueller:2012:ICI:2380116.2380191}.
While their system interprets objects as simple platonic geometry, our work takes the branches with diverse shapes.
While Crowdsourced fabrication project took advantage of humans-in-the-loop in their fabrication system \cite{lafreniere2016crowdsourced}, our work puts emphasis on crowd-sourced design.
As a crowd-sourced design system, Jerry et. al., developed a platform for light users to design trees and plants \cite{talton2009exploratory}.
Our design process is not parametric modeling, also directly linked to physical world. \todo{ill-logic}

There are few works that take natural materials with native shapes as design components.
Schindler and his colleagues used digitally scanned wood branches and used them for furniture and interior design elements \cite{schindler2014processing}.
Monier and colleagues virtually generated irregularly shaped branch-like components and explored designs of large scale structure ~\cite{monier2013use}.
Using larger shaped forked tree trunks, \textit{Wood Barn} project designed and fabricated custom joineries to construct a truss-like structure\cite{woodbarn}.
\textit{Smart Scrap} project digitally measured lime stone leftover slates from a quarry and digitally generated assembly pattern of slates \cite{smartscrap}.\\

In industry, recognition of irregularly shaped objects is essential for waste management.
\textit{ZenRobotics} developed a system that sorts construction and demolition waste by picking objects on a conveyor belt using robotic hands \cite{lukka2014zenrobotics}.
For factory automation purpose, there is a system that recognizes irregularly shaped objects and sort them into a container \cite{sujan2000design}.
Getting out from factories, autonomous robotics in construction site is a hot topic among roboticists \cite{feng2014towards}.
\textit{In-situ Fabricator} is a system which could be installed in construction site and co-operated with human workers \cite{dorfler2016mobile}.
Once robot is autonomously localize itself in such an environment, it can build foundational structure for further construction \cite{napp2014distributed}.
Using locally found objects on-site, such a system can be much simpler.\\

While these projects demonstrated the capability of digital fabrication processes to handle irregularly shaped materials, design process is still dependent on a designer or architect who has experiences with materials or has access to special software.

Cimerman discussed architectural design practices that took computer-mediated participatory design \cite{cimerman2000participatory}.
He mentioned three motivations of digital participatory design:
\begin{itemize}
 \item{Including stakeholders in creation of one's environment.}
 \item{Experimenting diverse design tastes from multiple point of views.}
 \item{Solving complex design tasks with full of diverse solutions.}
\end{itemize}
Opening database of available local materials, people with various backgrounds can involve in design process, which could lower the cost of design fee with natural materials with native shapes.
For example, \textit{Nano-Doc} took gamification approach to search valid nano-particle designs against tumors out of infinite design space \cite{hauertcrowdsourcing}.
\textit{DrawAFriend} has developed an online game to collect big-data for drawing applications which assist humans with auto-stroke assistance \cite{limpaecher2013real}.
While these works developed games which are collecting data for solving medical or engineering problems, our game intends to provide a participatory design platform as a solution for social and cultural problems context-aware architectural design with locally found natural materials.
 \todo{ambiguous!}
