\section{Related Work}
in architecture, buildings with pre-fabricated customized parts have been in real practices using 3D printers and CNC routers~\cite{knaack2012prefabricated}.
Researchers in graphics communities also worked on fabricating 3D complex shapes out of flat panels~\cite{schwartzburg2013fabrication,cignoni2014field,fu2015computational}.
Although they fabricated custom joints for custom outputs, their fabrication is ``workmanship with certainty'' due to the use of highly standardized materials and batch processing of G-Code, whereas our focus is fabricating joints on natural materials in native forms.

On the other hand, as ``workmanship with risks'' in digital fabrication, interactive fabrication enables machines to pick up uncertain happenings and react on it \cite{willis2011interactive}.
Mueller and her colleagues developed interactive laser cutting, taking user inputs and recognizing placed objects in a fabrication scene \cite{Mueller:2012:ICI:2380116.2380191}.
While their system interprets objects as simple platonic geometry, our work interacts with the native forms of irregularly shaped branches.
Crowdsourced Fabrication project took advantage of humans-in-the-loop in their fabrication system \cite{lafreniere2016crowdsourced}.
Similarly, additive manufacturing with human and a guidance system was developed \cite{Yoshida:2015:AHA:2809654.2766951}.
Unlike these works, our work puts emphasis on the involvement of humans in design.
As a crowdsourced design system, Talton and colleagues developed a platform for light users to design trees and plants \cite{talton2009exploratory}.
Our work also provides online collaborative design platform but our outputs are fabricated objects with materials in real-world.

There are few works that take natural materials in native forms as design components.
Schindler and his colleagues used digitally scanned wood branches and used them for furniture and interior design elements \cite{schindler2014processing}.
Monier and colleagues virtually generated irregularly shaped branch-like components and explored designs of large scale structure ~\cite{monier2013use}.
Using larger shaped forked tree trunks, \textit{Wood Barn} project designed and fabricated custom joints to construct a truss-like structure \cite{woodbarn}.
\textit{Smart Scrap} project digitally measured lime stone leftover slates from a quarry and digitally generated assembly pattern of slates \cite{smartscrap}.
An automated pick-and-stack process was explored with stones with irregular shapes. \cite{stonestacking}.
% In industry, recognition of irregularly shaped objects is essential for waste management.
% \textit{ZenRobotics} developed a system that sorts construction and demolition waste by picking objects on a conveyor belt using robotic hands \cite{lukka2014zenrobotics}.
% For factory automation purpose, there is a system that recognizes irregularly shaped objects and sort them into a container \cite{sujan2000design}.
% While these projects demonstrated the capability of digital fabrication processes to handle irregularly shaped materials, design process with native natural materials is still dependent on experts.

Cimerman discussed architectural design practices that took computer-mediated participatory (architectural) design \cite{cimerman2000participatory}.
He mentioned three motivations of digital participatory design:
1. including stakeholders in creation of one's environment.
2. experimenting diverse design tastes from multiple point of views.
3. solving complex design tasks with full of diverse solutions.

Database of available local materials allows people with various backgrounds to involve in design process, which could lower the design cost with natural materials in native forms.
We took inspiration from existing gamification systems.
For example, \textit{FoldIt} crowd-sourced protein structure prediction to game players ~\cite{cooper2010predicting}.
\textit{Nano-Doc} took gamification approach to search valid nano-particle designs against tumors out of infinite design space \cite{hauert2013computational}.
\textit{DrawAFriend} has developed an online game to collect big-data for drawing applications which assist humans with auto-stroke assistance \cite{limpaecher2013real}.
While these works developed games for collecting valuable data for solving medical or engineering problems, our game is served as a collaborative design platform, aiming to solve the socio-cultural issues in modern buildings such as generic design and detached context.
