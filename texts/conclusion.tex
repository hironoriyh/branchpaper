\section{Conclusion}
% \subsubsection*{Summary}
In this paper, we presented a workflow to design and fabricate with branches in their native forms, which are not large enough to be used as standardized building components.
Our workflow was validated by the case study with lower-aged participants without design and fabrication experiences.

Our online platform with stored scanned branches is accessible and multiple users can submit design layouts and explore a global design. % and locations.
Our branch joint detection and group condition update algorithms are running on the browser game which can be accessed from laptops and mobile devices, contributing to the accessibility of the presented workflow.
Together with the accessibility, the intuitive interface was simple for non-expert users, validated by the case study.
We successfully built a network of branches with rigid joints generated by our joint milling path generator.
Each of joints has customized lapped-joint geometry, which extends design possibilities of branches or woods with their native forms.

% \subsubsection*{Limitations and Future Work}
Our workflow consists of many technological components such as skeleton extraction, structural optimization, object detection/recognition, and data-driven design-fabrication.
Focusing on the use of native forms, each step of our workflow has potential to contribute to each area with the use of native forms of natural materials.
Also, our workflow was developed based on the participation of users, thus the entire process is not necessarily automated, however, some tasks could be improved to assist users.

The skeleton extraction could take incomplete point-set directly from original tree branches before they are cut in length.
With data-driven approach, the system could distinguish trees and which part of tree the branch from.
With morphological analysis, the system could suggest users where to cut branches to achieve user-defined target design.
Structural and geometrical validity/invalidity of obtained materials could be analyzed.
Our workflow requires branches to be fixed on a plate, which takes the longest duration in the workflow except for in-between tasks such as moving and pausing.
Using a robotic manipulator with a gripper, the attaching process could be skipped.

Our game system is limited in 2D, whereas original branch forms have rich 3D geometry with textures.
In our case, these information was used in limited ways such as in skeleton extractions and G-Code generation.
Despite of successfully fabricated non-orthogonal joints, we did not complete the attaching branches to the target frames, as we prioritized to validate branch-branch joints.
Our layout design process is fully dependent on users with limited feedback during design process.
The game can provide suggestive feedback with structural analysis of each joint and entire structure.
% As the problem of limited number of branches and fixed target points to be connected, the system could assist humans to reach to structurally sound solutions with less efforts.

% The performance of our joint detection and alalgorithm could be
Our joint and group detection algorithms are limited with materials with skeletons, and our joint generator is limited to branches.
Both steps use down-sampled or high-resolution point sets.
It is valuable to validate the approach by comparing with other available methods such as collision detections or joint detection with down-sampled model by interpolation.

Finally, our game-based design could be applied to different purposes, not only for participatory layout design but also for collecting data of user behaviors during design.
Also application to other kinds of materials could be investigated.
