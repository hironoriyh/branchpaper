\section{Conclusion}
\subsubsection*{Summary}
In this paper, we presented a workflow to design and fabricate with branches with their native forms, which are not large enough for producing standardized building components.
Our workflow was validated by the case study with lower-aged participants without design and fabrication experiences.
Our online platform with stored scanned branches is accessible and multiple users can submit design layouts and explore a global design. % and locations.
Our branch joint detection and group condition update algorithms are running on the browser game which can be accessed from laptops and mobile devices, contributing to the accessibility of the presented workflow.
Together with the accessibility, the intuitive interface was simple for non-expert users, validated by the case study.
We successfully built a network of branches with rigid joints generated by our joinery milling path generator.
Each of joints has customized lapped-joint geometry, which extends design possibilities of branches or woods with their native forms.

\subsubsection*{Limitations and Future Work}
Our game system is limited in 2D, whereas original branch forms have rich 3D geometry with textures.
In our case, these information was used in limited ways such as in skeleton extractions and G-Code generation.
Our workflow requires branches to be fixed on a plate, which takes the longest duration in the workflow except for in-between tasks such as moving and pausing.
Despite of successfully fabricated non-orthogonal joineries, we did not complete the attaching branches to the target frames, as we prioritized to validate branch-branch joineries.

Our design workflow fully dependent on human's ability for obtaining design solutions.
As the problem of limited number of branches and fixed target points to be connected, the system could assist humans to reach to structurally sounding solutions with less efforts.
Collected data from game could be further analyzed for extracting meaningful data for the development of assisting algorithm.
% The performance of our joint detection and alalgorithm could be
Our joint and group detection, and milling path algorithms are not applicable to other purposes at this moment.
It is valuable to compare the performance of our joint and group detection system to other collision detection library.
In our workflow, the pipeline is not seamlessly connected: experienced operators need to take care of retouching mesh model, Branch Importer, and G-Code generator, and operating a CNC router.
