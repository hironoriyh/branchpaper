\section{Conclusion}
In this paper, we presented a workflow to design and fabricate with branches which are not big enough for construct structural applications.
Our workflow was validated with case-study with lower-aged participants without design and fabrication experiences.
Our online platform to store geometries, designs was reachable by people with various backgrounds and locations.
Our light-weight branch joint detection algorithm contributed to the accessibility of the platform, and the intuitive interface was simple enough for non-expert users.
We successfully built a network of branches with rigid joints generated by our joinery milling path generator.
Each of joints has customized lapped-joint geometry, which is difficult to design and fabricate by even experts.

Limitations: \\
Our system is limited in 2D, separated frames, etc.
Acquired data from game was not fully processed for extracting meaningful applications.
Our joint and group detection, and milling path algorithms are not generalizable.
The pipeline is not seamlessly connected: experienced operators need to take care of retouching mesh model, Branch Importer, and G-Code generator, and operating a CNC router.
