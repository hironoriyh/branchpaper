5.2.1 Project overview
Urban open spaces such as parks in cities produce tons of branches every year, and most of them are cut and burned. As an experimental design fabrication project seeking to maximize the material resources of a city, a participatory digital design fabrication system is presented for organizing irregular shaped branches. Although design as well as evaluation could be implemented by an optimization algorithm or by experienced designers and engineers, the proposed approach is participatory design through an online game (BranchConnect), collecting numerous solutions from people playing the game. Branches are physically collected, scanned, and stored in a digital library. The game has a scoring system maximizing structural properties with given branches. The game collects a number of organization solutions and the solutions on the top of the high score list are used for real production; they are CNC milled and connected with a uniquely developed joint system that adjusts to the various thicknesses and shapes of the branches. 

5.2.2 Objective
Since there are millions of combinations of branch organizations, and moreover the created complex connections cannot be standardized for structural analysis, it is computationally expensive to optimize the structure. A different approach is illustrated by the example of NanoDocs, which optimized the molecular structure of cancer medicine through a game in which many players participated and pursued the best combination of atoms (Hauert, 2013). Similarly, the project developed a game so that many participants could be involved in the design process and search for the best structural organization. Collecting as many solutions as possible from among numerous participants, the problem of the structure could be solved by creative users instead of through expensive computation. 

5.2.3 Workflow
a. Scanning and skeletonization of branches
The obtained point cloud provides contours of branches as well as diameters. The diameter of a branch is calculated by taking median value of distances from among contour vertexes of the branch. The height value simply takes the depth value of the branch, however, if a branch is twisted in 3D with undercut, the depth value and the diameter are used to acquire the top height as well as the bottom height. Through point cloud processing and skeleton extraction, branches are categorized into 3 types: I, V, and Y shapes. 

b. The game: BranchConnect
The developed game interface is composed of two elements: 1. a frame to be filled by the branches, and 2. a set of available branches. Users need to connect two given points on the frame by taking and connecting the branches from among the available set to reinforce the structure. Since the CNC fabrication process can tool only the top side of the branches, a branch of all the connected pair needs to be flipped. Evaluating the connection, it takes skeletons of the scanned branches and calculates the closest point among the skeletons. Each branch has a set of variables; diameter, length and shape, hence the set of available branches can be sorted in preferential order.

c.  Fabrication
Taking the location and rotation data together with contour points, motion paths for CNC milling are calculated and exported in G-code. A customized halved joint locking rotation and position was developed (Seike, 1977). 





BranchConnect
A.2.1 Project overview
Urban open spaces such as parks in cities produce tons of branches every year, and most of them are chipped and burned. As an experimental design fabrication project maximizing the material resources of a city, a participatory digital design fabrication system is presented for organizing irregular shaped branches. Although design as well as evaluation could be implemented by an optimization algorithm or by experienced designers and engineers, the proposed approach is participatory design through an online game (BranchConnect), collecting numerous solutions from people playing the game. Branches are physically collected, scanned, and stored in a digital library. The game has a scoring system maximizing structural properties with given branches. The game collects a number of organization solutions and the solutions on the top of the high score list are used for real production; they are CNC milled and connected with a uniquely developed joint system that adjust to the various thicknesses and shapes of the branches. 

A.2.2 Objective
Architecture has embraced the direct use of natural materials, for instance a facade consisting of uncountable shingles, and curved timbers as structural beams in farmer's house. The complexity in fabrication between lines and curves are cancelled by digital fabrication tools, therefore the direct use of natural materials cuts off material processing for standardization, resulting in lower cost, smaller amount of energy consumed. On the other hand, urban open spaces produces tons of branches. For example, the Hongo campus of University of Tokyo (540000㎡) annually produces 70-100 tons of branches (Facility maintenance department in the U-Tokyo). However, the resource is not opened to the public, and most are chipped and burned. The challenge here is to make use of branches, which usually thrown away and burnt, for building, interior and furniture components.
	Since there are millions of combination of branch organizations, and moreover the complex connection cannot be standardized for structural analysis, it is computationally expensive to optimize the structure. A different approach is to do like for example NanoDocs, which optimized the molecular structure of cancer medicine through a game in which many players participated and pursued the best combination of atoms (Hauert, 2013). Similarly, the project developed a game so that many participants can involve in the design process and seek for the best structural organization. Collecting as many solutions as possible among numerous participants, the ill-defined problem of the structure could be solved by creative users instead of by expensive computation. 

A.2.3 Workflow
a. Scanning and skeltonization of branches
300 branches with around 500 mm length, 50-100 mm diameter were collected from the campus of University of Tokyo. Branches were cut in proper length and 5-8 were laid on a board to scan them all at once. A laser range finder (URG-40, Hokuyo) attached on a 3 axis CNC was used for acquiring 3D shapes of branches by slit scanning method. 
	The obtained point cloud provides contours of branches as well as diameters. The diameter of a branch is calculated by taking median value of distances among contour vertex of the branch. Height value simply takes the depth value of the branch, however in case some branches are twisted in 3D with undercut, the depth value and the diameter are used to acquire the top height as well as the bottom height. Through point cloud processing and skeleton extraction, branches are categorized into 3 types: I, V and Y shape. I shape has a straight continuous curve, V has an inflection point, and Y has a splitting point. I and V are distinguished by the intensity of angle between two points next each other. Y shape recognition first segmented the contours into triangles by Delaunay triangulation, and then picked a middle point of each triangle line (Igarashi et al., 1998). In case of Y shape, connecting each middle point, it creates two curves which are used for detecting Y shape. 

b. The game: BranchConnect
The developed game interface are twofold: 1. a frame to be filled by the branches, and 2 a set of available branches. Users need to connect two given points on the frame by taking and connecting the branches among the available set to reinforce the structure. Since the CNC fabrication process can tool only top side of the branches, a branch of all the connected pair needs to be flipped. Evaluating the connection, it takes skeletons of the scanned branches and calculating the closest point among the skeletons. Each branch has a set of variables; diameter, length and shapes, hence the set of available branches can be sorted as preferable order. 
	
		Scoring system is designed to maximize the given structure with minimum use of branches. Minimum use is calculated by the total length of the connected branches. The structure is evaluated comparing the distance of the given two points and the length of connected branches (Figure A.6). 
	The assembly is evaluated according to several criteria giving plus and mi-nus points. All types of joints and connections give plus points, whereas instances of "invalid overlaps" (when branches are intersecting without making a valid joint) and "islands" (when a branch or group of branches lack connection to the main assembly) and the length of the parts of the branches that needs to be cut of when making a boundary joint or a linear joint give minus points. The sum of the score forms the basis for raking the assemblies. 
		The user can chose to start from scratch, or to pick up an existing design and improve it. If he or she manages to make a rearrangement that increases the score – the assembly will be updated, and he or she will be honored as contributor listed in order according to how many points they managed to add to the previous branch organization. In this way, rather than everyone competing against each other, the assembly designs evolve and become perfected over time, with the help of many users. And, if the only option was the start a new design, the new user might be discouraged thinking that he or she can never achieve as high of a score as the experienced users. But to add something to an existing organization is a doable challenge even for the new user, encouraging participation by many.
	When all the required points are connected by branches and the user satisfied the pattern, the game is completed. The location and rotation of branches will be passed to the fabrication algorithm.




c. Fabrication
Taking the location and rotation data together with contour points, motion paths for CNC milling are calculated and exported in G-code. A customized halved joint locking rotation and position was developed (Seike, 1977). Unlike ordinal halved joint, due to the round section of branches, surfaces on sides offset from a contour has to be plane, together with the overlapped area. To fabricate the developed joinery system by CNC milling, two different kind of milling paths, side-cuts for the side surfaces and center-cut for the overlapped area, are exported for each joinery. Each joinery modeled parametrically based on angle of connection (up to 45 degree) and diameter of each branch.  