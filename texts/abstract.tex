\begin{abstract}
Diverse natural materials such as stones and woods have been used as architectural elements preserving their native forms since primitive shelters, however, the use of them in modern buildings is limited due to the irregular properties.
In this paper, we take the diversity as playful inputs for design task, and present our participatory game-based design-fabrication platform for customized architectural elements.
Taking tree branches as case study of materials with native forms, our online game \textit{BranchConnect} enables users to design 2D networks of branches and realize the design.
As the generated design is realizable with ordinal 3-axis CNC milling machines, each connection has a customized unique joinery adapted to the native branch shapes.
The scoring system of the game guides users to design structurally sound solutions with given branches.
Together with low-cost mobile scanning devices, users with diverse contexts can contribute to design and fabrication process not only as a game user, but also from collecting branches around their physical environments and uploading them to our online platform.
For validating our process, we conducted a workshop with end-users (children and their parents).
They collected branches in a nearby forest and contributed to design/fabricate a 2D fence with our system.
\end{abstract}
