\begin{abstract}
Diverse natural materials such as stones and woods have been used as architectural elements preserving their native forms since primitive shelters, however, the use of them in modern buildings is limited due to their irregular properties.
In this paper, we take the diversity as playful inputs for design task, and present our game-based design-fabrication platform for customized architectural elements.
Taking tree branches as a material with native forms, a game \textit{BranchConnect} enables end users to design 2D networks of branches and fabricate it by a CNC router.
The game considers fabrication constraints such as limitations with ordinal 3-axis CNC routers.
Each connection has a customized unique joinery adapted to the native forms of branches.
The scoring system of the game guides users to design structurally sound solutions with given branches.
Together with low-cost mobile scanning devices, users with diverse contexts can contribute to design and fabrication process not only by playing the game, but also by collecting branches around their physical environments and uploading them to our online platform.
For validating our process, we conducted a workshop with children and their parents from a local community.
They collected branches in a nearby forest and contributed to design and fabricate a 2D fence with our system.
\end{abstract}
